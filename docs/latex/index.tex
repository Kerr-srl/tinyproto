This is Tiny protocol implementation for microcontrollers (Arduino, Stellaris).\hypertarget{index_introduction}{}\section{Introduction}\label{index_introduction}
This protocol is intended to be used in low-\/memory systems, like microcontrollers (Stellaris, Arduino). It is also can be compiled for desktop Linux systems, and it is possible to build it for Windows.\hypertarget{index_api}{}\section{Tiny Protocol A\+PI}\label{index_api}
Linux A\+PI supports C-\/\+Style functions, while Arduino A\+PI supports both C-\/\+Style and C++ classes. Usage C++ Arduino classes makes easy use of the protocol for Arduino projects. Please refer to documentation.\hypertarget{index_packet}{}\section{Packet Structure}\label{index_packet}
Full packet format\+: 
\begin{DoxyPre}
     8        any len    None/8/16/32     8
 |   7E   |  USER DATA  |    FCS     |   7E   |
\end{DoxyPre}


Full duplex hdlc uses standard hdlc frame format\+: 
\begin{DoxyPre}
     8     ADDR  CONTROL     any len    None/8/16/32     8
 |   7E   | FF | hdlc ctl | USER DATA  |    FCS     |   7E   |
\end{DoxyPre}



\begin{DoxyItemize}
\item 7E is packet delimiter (commonly used on layer 2) as defined in H\+D\+L\+C/\+P\+PP framing. packet delimiter is used by the protocol to find first and last byte of the frame being transmitted.
\item U\+S\+ER D\+A\+TA of any length. This field contains user data with replaced 0x7D and 0x7E bytes by special sequenced as defined in H\+D\+L\+C/\+P\+PP framing. Length of data is now limited on buffer used to receive frames and/or 32767 bytes (Tiny Protocol using 16-\/bit field to store frame length).
\item F\+CS is standard checksum. Depending on your selection, this is 8-\/bit, 16-\/bit or 32-\/bit field, or it can be absent at all. Refer to R\+F\+C1662 for examples. F\+CS field is also optional and can be disabled. But in this case transport errors are not detected.
\end{DoxyItemize}\hypertarget{index_callback}{}\section{User-\/defined callbacks}\label{index_callback}
\hyperlink{group__HDLC__API}{Tiny H\+D\+LC protocol A\+PI functions} needs 3 callback functions, defined by a user (you may use any function names you need).

H\+D\+LC callbacks\+: 
\begin{DoxyCode}
\textcolor{keywordtype}{int} write\_func\_cb(\textcolor{keywordtype}{void} *user\_data, \textcolor{keyword}{const} \textcolor{keywordtype}{void} *data, \textcolor{keywordtype}{int} len);
\textcolor{keywordtype}{int} on\_frame\_read(\textcolor{keywordtype}{void} *user\_data, \textcolor{keywordtype}{void} *data, \textcolor{keywordtype}{int} len);
\textcolor{keywordtype}{int} on\_frame\_sent(\textcolor{keywordtype}{void} *user\_data, \textcolor{keyword}{const} \textcolor{keywordtype}{void} *data, \textcolor{keywordtype}{int} len);
\end{DoxyCode}



\begin{DoxyItemize}
\item write\+\_\+func\+\_\+cb() is called by H\+D\+LC implementation every time, it needs to send bytes to TX channel
\item on\+\_\+frame\+\_\+read() is called by H\+D\+LC implementation every time, new frame arrives and checksum is correct.
\item on\+\_\+frame\+\_\+sent() is called by H\+D\+LC implementation every time, new frame is sent to TX. H\+D\+LC protocol requires only write\+\_\+func\+\_\+cb() to be defined. Other callbacks are optional. As for RX processes, your application code is responsible for reading data from RX line, then all you need to do, is to pass received bytes to H\+D\+LC implementation for processing via \hyperlink{group__HDLC__API_ga911a3f1cb32dd6cadd00223e0097642c}{hdlc\+\_\+run\+\_\+rx()}.
\end{DoxyItemize}

All higher level protocols (\hyperlink{group__LIGHT__API}{Tiny light protocol A\+PI functions}, \hyperlink{group__HALF__DUPLEX__API}{Tiny Half Duplex A\+PI functions}) needs 4 callback functions, defined by a user\+: read\+\_\+func\+\_\+cb() is added. The list of callbacks\+:


\begin{DoxyCode}
\textcolor{keywordtype}{int} write\_func\_cb(\textcolor{keywordtype}{void} *user\_data, \textcolor{keyword}{const} \textcolor{keywordtype}{void} *data, \textcolor{keywordtype}{int} len);
\textcolor{keywordtype}{int} read\_func\_cb(\textcolor{keywordtype}{void} *user\_data, \textcolor{keywordtype}{void} *data, \textcolor{keywordtype}{int} len);
\textcolor{keywordtype}{int} on\_frame\_read(\textcolor{keywordtype}{void} *user\_data, \textcolor{keywordtype}{void} *data, \textcolor{keywordtype}{int} len);
\textcolor{keywordtype}{int} on\_frame\_sent(\textcolor{keywordtype}{void} *user\_data, \textcolor{keyword}{const} \textcolor{keywordtype}{void} *data, \textcolor{keywordtype}{int} len);
\end{DoxyCode}


Unlike H\+D\+LC implementation, higher level protocols use different approach. They control both TX and RX channels, for example, to transparently send A\+CK frames, etc. That\textquotesingle{}s why higher level protocols need to read\+\_\+func\+\_\+cb to be defined\+:


\begin{DoxyItemize}
\item read\+\_\+func\+\_\+cb() is called by higher level protocol implementation every time, it needs to read bytes from RX channel.
\end{DoxyItemize}\hypertarget{index_arduino_section}{}\section{Arduino related documentation}\label{index_arduino_section}
\hyperlink{arduino}{Arduino related A\+PI (C++)} 