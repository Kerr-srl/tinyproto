This is Tiny protocol implementation for microcontrollers (Arduino, Stellaris).\hypertarget{index_introduction}{}\section{Introduction}\label{index_introduction}
This protocol is intended to be used in low-\/memory systems, like microcontrollers (Stellaris, Arduino). It is also can be compiled for desktop Linux systems, and it is possible to build it for Windows.\hypertarget{index_api}{}\section{Tiny Protocol A\+PI}\label{index_api}
Linux A\+PI supports C-\/\+Style functions, while Arduino A\+PI supports both C-\/\+Style and C++ classes. Usage C++ Arduino classes makes easy use of the protocol for Arduino projects. Please refer to documentation.\hypertarget{index_packet}{}\section{Packet Structure}\label{index_packet}
Full packet format\+: 
\begin{DoxyPre}
     8       16       any len    None/8/16/32     8
 |   7E   |  UID  |  USER DATA  |    FCS     |   7E   |
\end{DoxyPre}



\begin{DoxyItemize}
\item 7E is packet delimiter (commonly used on layer 2) as defined in H\+D\+L\+C/\+P\+PP framing. packet delimiter is used by the protocol to find first and last byte of the frame being transmitted.
\item U\+ID mean unique identifier of the frame. For Half Duplex Tiny Protocol (H\+A\+L\+F\+\_\+\+D\+U\+P\+L\+E\+X\+\_\+\+A\+PI) it is widely used in acknowledge frames. This field is optional, you may omit U\+ID parameter in S\+I\+M\+P\+L\+E\+\_\+\+A\+PI and A\+D\+V\+A\+N\+C\+E\+D\+\_\+\+A\+PI functions.
\item U\+S\+ER D\+A\+TA of any length. This field contains user data with replaced 0x7D and 0x7E bytes by special sequenced as defined in H\+D\+L\+C/\+P\+PP framing. Length of data is now limited on buffer used to receive frames and/or 32767 bytes (Tiny Protocol using 16-\/bit field to store frame length).
\item F\+CS is standard checksum. Depending on your selection, this is 8-\/bit, 16-\/bit or 32-\/bit field, or it can be absent at all. Refer to R\+F\+C1662 for examples. F\+CS field is also optional and can be disabled. But in this case transport errors are not detected.
\end{DoxyItemize}\hypertarget{index_arduino_section}{}\section{Arduino related documentation}\label{index_arduino_section}
\mbox{\hyperlink{arduino}{Arduino related A\+PI (C++)}} 