This is Tiny protocol implementation for microcontrollers (Arduino, Stellaris).

\section*{Tiny Protocol}

\subsection*{Introduction}

This protocol is intended to be used in low-\/memory systems, like microcontrollers (Stellaris, Arduino). It is also can be compiled for desktop Linux systems, and it is possible to build it for Windows.

\subsection*{Tiny Protocol A\+P\+I.}

Linux A\+P\+I supports C-\/\+Style functions, while Arduino A\+P\+I supports both C-\/\+Style and C++ classes. Usage C++ Arduino classes makes easy use of the protocol for Arduino projects. Please refer to documentation.

\subsection*{Tiny Protocol Packet structure.}

Full packet format\+: 
\begin{DoxyPre}
   8    16     any len     8/16/32  8
 | 7E | UID |  USER DATA  |  FCS  | 7E |
\end{DoxyPre}



\begin{DoxyItemize}
\item 7\+E is standard packet delimiter (commonly used on layer 2) as defined in H\+D\+L\+C/\+P\+P\+P framing.
\item U\+I\+D can be used for your own needs. This field is intented for use with higher levels protocols. This field is optional, you may omit U\+I\+D parameter in Tiny Protocol A\+P\+I functions.
\item U\+S\+E\+R D\+A\+T\+A of any length
\item F\+C\+S is standard checksum. Depending on your selection, this is 8-\/bit, 16-\/bit or 32-\/bit field. Refer to R\+F\+C1662 for examples. F\+C\+S field is also optional and can be disabled. But in this case transport errors are not detected.
\end{DoxyItemize}

Each packet can be sent with simple api command\+: 
\begin{DoxyCode}
\textcolor{keywordtype}{int} \hyperlink{src_2arduino_2src_2proto_2tiny__layer2_8h_a988a41addbe75dc15cc13006de6740e0}{tiny\_send}(SHdlcData *handle, uint16\_t *uid, uint8\_t *pbuf, \textcolor{keywordtype}{int} len, uint32\_t flags)
\end{DoxyCode}


And be received via another simple api command\+:


\begin{DoxyCode}
\textcolor{keywordtype}{int} \hyperlink{src_2arduino_2src_2proto_2tiny__layer2_8h_a470d59a60a496944e63031ba43a00e3d}{tiny\_read}(SHdlcData *handle, uint16\_t *uid, uint8\_t *pbuf, \textcolor{keywordtype}{int} len, uint32\_t flags);
\end{DoxyCode}


uid parameter is optional and can be be simply N\+U\+L\+L. And F\+C\+S field can be disabled by calling 
\begin{DoxyCode}
\hyperlink{src_2arduino_2src_2proto_2tiny__layer2_8h_a5e66725a2818491d4e2b1134951d9229}{tiny\_set\_fcs\_bits}(&handle, 0);
\end{DoxyCode}
 In this case, U\+I\+D field and F\+C\+S field are not used, and the protocol packet looks as follows\+: 8 any len 8 $\vert$ 7\+E $\vert$ U\+S\+E\+R D\+A\+T\+A $\vert$ 7\+E $\vert$

So, the protocol itself adds a little overhead to the user data\+: 2 bytes of packet delimiter in the simplest case. And in the case of full packet format (both with U\+I\+D and 32-\/bit F\+C\+S fields) the protocol adds 8 bytes of overhead. 